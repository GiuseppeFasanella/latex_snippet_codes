\documentclass{standalone}
\usepackage{tikz}
\usetikzlibrary{positioning}

\begin{document}
\begin{tikzpicture}

% definizione delle caratteristiche del nodo
\tikzset{mynode/.style={circle,draw, minimum width=0.1cm, minimum height=0.1cm}}
% definizione colore di riempimento  
\tikzset{myfillcolor/.style ={fill=#1, draw=none}}
%Traccia gli archi
\draw (-4,0) arc [start angle=180, end angle=-180, x radius=4, y radius=4];
\draw (-8,0) arc [start angle=180, end angle=-180, x radius=8, y radius=8];

%Disegna i nodi sugli archi
\foreach \a in {1,2,...,12}{
\node[mynode,myfillcolor=black] at (\a*360/12: 4 cm) {};
\draw (\a*360/12: 4 cm)--(\a*360/12: 8 cm);
}
\foreach \a in {1,2,...,12}{
\node[mynode,myfillcolor=black] at (\a*360/12: 8 cm) {};
}


%Nomi delle tonalita' maggiori
\draw (3*360/12: 9 cm) node{Do +};
\draw (2*360/12: 9 cm) node{Sol +};
\draw (1*360/12: 9 cm) node{Re +};
\draw (0*360/12: 9 cm) node{La +};
\draw (-1*360/12: 9 cm) node{Mi +};
\draw (-2*360/12: 9 cm) node{Si +};
\draw (-3*360/12: 8.5 cm) node{Fa\# +};
\draw (-3*360/12: 9 cm) node{(Solb +)};
\draw (-4*360/12: 9 cm) node{Reb +};
\draw (-5*360/12: 9 cm) node{Lab +};
\draw (-6*360/12: 9 cm) node{Mib +};
\draw (-7*360/12: 9 cm) node{Sib +};
\draw (-8*360/12: 9 cm) node{Fa +};
%Nomi delle tonalita' minori
\draw (3*360/12: 3 cm) node{La -};
\draw (2*360/12: 3 cm) node{Mi -};
\draw (1*360/12: 3 cm) node{Si -};
\draw (0*360/12: 3 cm) node{Fa\# -};
\draw (-1*360/12: 3 cm) node{Do\# -};
\draw (-2*360/12: 3 cm) node{Sol\# -};
\draw (-3*360/12: 2.5 cm) node{Re\# -};
\draw (-3*360/12: 3 cm) node{(Mib -)};
\draw (-4*360/12: 3 cm) node{Sib -};
\draw (-5*360/12: 3 cm) node{Fa -};
\draw (-6*360/12: 3 cm) node{Do -};
\draw (-7*360/12: 3 cm) node{Sol -};
\draw (-8*360/12: 3 cm) node{Re -};



\end{tikzpicture}
\end{document}
