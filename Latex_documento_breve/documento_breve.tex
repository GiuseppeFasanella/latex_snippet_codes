\documentclass[a4paper,11pt]{article}
\usepackage[T1]{fontenc}
\usepackage[utf8]{inputenc}
\usepackage[a4paper,top=4.0cm,bottom=4.5cm,left=3.5cm,right=3.5cm,
bindingoffset=0mm]{geometry}
\usepackage[italian]{babel}
\usepackage{amsmath}
\usepackage{graphics}
\usepackage{graphicx}
\usepackage{textcomp}
\usepackage{array}
\usepackage{tabularx}
\usepackage{booktabs}
\usepackage{multicol}
\usepackage{listings}
\lstset{language=C++,
basicstyle=\small\ttfamily,
showstringspaces=false,
numbers=left,
numberstyle=\tiny,
breaklines=true,
columns=fullflexible}
\usepackage{hyperref}
\usepackage{amsmath}
\usepackage{amssymb}
\usepackage{fancyhdr}
\usepackage{comment}
\usepackage{feynmf}
\pagestyle{fancy}
\lhead{}
\rhead{\small \slshape \leftmark}




\begin{document}

\begin{center}
\Large{\textbf{Search for new physics at the LHC with the CMS detector}}
\end{center}

\begin{center}
 Ph.D. Research Project
\end{center}

\begin{center}
\textbf{Ph.D. Candidate:} Giuseppe Fasanella
\newline
\textbf{Supervisor:} Dr. Paolo Meridiani (Sapienza Università di Roma)
\newline
\textbf{Supervisor:} Prof. Barbara Clearbaux (Université Libre de Bruxelles)
\newline
\newline
\small{Sapienza Università di Roma. Dottorato in Fisica, XXIX Ciclo}
\end{center}

\bigskip
\large{\textbf{Introduction}}

\smallskip
The Standard Model (SM) is an elegant theory, which describes the fundamental interactions among particles. %[1]. 
Up to now, its success in reproducing a large amount of experimental data, spanning several orders of magnitude in energy, is impressive.
The final piece of the model, the elementary scalar boson introduced in 1964 by Brout, Englert  and Higgs %[2,3]
has been recently discovered at a mass of around 125 GeV, by the  ATLAS and CMS experiments, 48 years after its prediction. This discovery is one of the best achievements of modern particle physics research. Nevertheless, despite its success, 
the SM cannot answer several fundamental questions (coupling unification, the dark matter candidate, matter-antimatter 
asymmetry...). Moreover, the so-called \textit{hierarchy problem} raises doubts about the naturality of the SM theory, which demands for a
huge level of fine-tuning in the cancellation of its divergencies.
For these reasons, even if the SM is renormalizable and able to produce predictions at any order in perturbation theory, it is
generally considered as a low energy effective model of a more fundamental, though still unkown, theory.

\medskip
\large{\textbf{Research project}} 

\smallskip
My research project is based on a co-supervision agreement between ``La Sapienza'' University of Rome and 
the ``ULB''- Université Libre de Bruxelles.
The goal of my research is to contribute to the search for new particles at the LHC in two domains:
(i) the search for new heavy neutral resonances (Z’ or G) decaying into electron-positron 
pairs. %It is worth noting that the masses of these new particles (and also their couplings, in some cases) are not constrained in the models, so they are free parameters to be determined ; 
(ii) the search for vector-like partners of the top quark T. 
%This search is performed using the proton-proton collision data recorded in 2012 by the CMS experiment at a center of mass energy of 8 TeV 

Both of these possibilities are foreseen in several Beyond Standard Model (BSM) models, which have been proposed during the last decades, in order to 
 address some of the SM problems mentioned above.
On one hand, new neutral bosons heavier than the SM Z boson, generically called Z' bosons, arise for example from Great Unification Theories (GUT) 
and models introducing extra-dimensions (ED). %[9]. 
\newline
On the other hand, other classes of new models, in particular composite Higgs models,  %[11]
introduce new ``strong'' dynamics in the framework and propose the existence of new particles, vector-like partners of the standard quarks.
In particular, vector-like replicas of the top quark, generally called T, are required to avoid the problem of the scalar boson mass divergence in the theory.

\smallskip
In addition, since the electromagnetic calorimeter (ECAL) of the CMS experiment is the main detector used in the analyses I'm working on,
I also plan to dedicate part of my work in order to address some specific problems of the ECAL detector, in particular the ECAL calibration at high energy.

\medskip
\large{\textbf{Search for heavy neutral resonances decaying in a dielectron pair}}

\smallskip
The search for heavy resonances decaying in electron-positron pairs has been perfomed at the LHC, during the Run1, obtaining a statement of exclusion of
a Z' particle up to 2-2.5 TeV, depending on the particular model under consideration.
With the LHC Run2, the increase in the center of mass energy of a factor $\approx$1.5 with respect to Run1, offers the opportunity to increase the sensibility 
of this analysis up to ~4 TeV with few statistics needed.
I will work in the HEEP group of the ``Université Libre de Bruxelles'', with the aim of studying final states with an electron-positron pair, 
coming from a new heavy neutral boson (Z'). The production of such a new particle could be detected by observing a peak (or a deformation),
compared to the SM prediction, in the invariant mass spectrum of the two electrons $M_{ee}$ at values above the TeV scale (above the limits obtained 
with the RUN1 dataset analysis).
In the coming months, I will contribute to the definition and the optimisation of the new HEEP selection for the high energy run.
The main aim is to be ready to analyse the new data collected at CMS, as soon as they arrive starting from spring/summer 2015, having previously tested 
the whole analysis chain on simulations. 

\medskip
\large{\textbf{Search for vector-like replicas (T) of the top quark}}

\smallskip
After few months of data-taking, as soon as the integrated luminosity increase at the order of $\approx$ 10 $\textrm{fb}^{-1}$, it would be also possible to
perform a search for new vector-like replicas of the top quark, in the presence of a scalar boson 
decaying in two photons. 
For this purpose, I will work in collaboration with the CMS group of “La Sapienza”-University of Rome.
In particular, the decay of the T particle in a standard model top and a Higgs boson ( T$\rightarrow$ tH) will be considered.
In addition to the T pair production, which has been studied at the LHC Run1, I will also study the single T production, more model dependent, but
becoming dominant, in several scenarios, with respect to the pair production mode at high T masses (above 900 GeV).
 
\begin{thebibliography}{9}
\bibitem{GUT}
W. de Boer, \textit{Grand Unified Theories and Supersymmetry in Particle Physics and Cosmology} hep-ph/9402266
\label{GUT}
\bibitem{ED}
Thomas G. Rizzo, \textit{Introduction to extra-dimension} arxiv:1003.1698
\label{ED}
\end{thebibliography}

\end{document}
