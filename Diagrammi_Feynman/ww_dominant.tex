
%% LA REGOLA CHE LUI USA PER NUMERARE E' METTERE I NUMERI PIU' BASSI SOTTO
%PERCIO' LA PARTICELLA o1 sara' quella piu' in basso e salendo metterà o2, o3,o4.....
\documentclass{article}
\usepackage{feynmp}
\usepackage{color}
\pagestyle{empty} %Takes away pagenumbers etc.
\begin{document}
\unitlength = 1mm
\begin{fmffile}{ww_dominant_diag} %Creating a diagram called "output"
 \begin{fmfchar*}(40,25)
  \fmfleft{i1,i2} \fmflabel{$\bar{q}$}{i1} \fmflabel{$q$}{i2}
  \fmf{plain}{i2,w2,w1,i1}
  \fmf{boson,label=$W^-$,l.s=right}{w1,v1}
  \fmf{boson,label=$W^+$,l.s=left}{w2,v2}
  \fmf{plain}{o4,v2,o3}
  \fmf{plain}{o2,v1,o1}
  \fmfright{o1,o2,o3,o4} \fmflabel{$\nu$}{o1}\fmflabel{\color{red}$l^{-}$}{o2} \fmflabel{$\nu$}{o3}\fmflabel{\color{red}$l^{+}$}{o4}
 \end{fmfchar*}
\end{fmffile}
\end{document}
%