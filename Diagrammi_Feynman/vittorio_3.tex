\documentclass{article}
\usepackage{feynmp}
\pagestyle{empty} %Takes away pagenumbers etc.
\begin{document}

\unitlength = 1mm
\begin{fmffile}{vittorio_3_diag} %Creating a diagram called "my_process_diag"
\begin{fmfchar*}(40,25)%(x.y) sono la grandezza del diagramma
%Particelle stato iniziale(i) e finale (o)
\fmfleft{i1,i2} \fmflabel{$g$}{i1}\fmflabel{$g$}{i2}
\fmfright{o1,o2} \fmflabel{$q$}{o1} \fmflabel{$\bar{q}$}{o2} 
% ||^ particelle stato finale, con label. Potresti farlo anche al momento della costruzione delle linee, ma 
% in questo modo i label sono molto meglio piazzati

%comincia a chiudere le linee e costruire il diagramma
\fmf{gluon}{i1,v1} %c'è un gluone che da i1 passa per il vertice v1 e va in i2 (quindi una forma a > nello stato iniziale
\fmf{gluon}{i2,v2} %c'è un gluone che fa il propagatore tra i vertici v1 e v2
\fmf{plain}{o1,v1,v2,o2}

%\fmfdot{v1,v2}% questo decide quali vertici abbiamo un punto marcato
\end{fmfchar*}
\end{fmffile}









\end{document}