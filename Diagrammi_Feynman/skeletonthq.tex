\documentclass{article}
\usepackage{feynmp}
\usepackage{color}
\pagestyle{empty} %Takes away pagenumbers etc.
\begin{document}

\unitlength = 1mm
\begin{fmffile}{skeletonthq_diag} 
\begin{fmfgraph}(40,40)
\fmfstraight
\fmfleft{i1,i2}%i0,i1,i2,i3,i4 e funzionava
\fmfright{o1,o2,o3,o4,o5}
\fmf{fermion,tension=2.5}{i1,v1}
\fmf{fermion,tension=1.5}{v1,vb}
\fmf{fermion}{vb,o1}%con o1 funzionava
\fmf{boson}{vb,o2}
\fmf{fermion,tension=1.5}{i2,v2}%,tension=1.5
\fmf{fermion,tension=1.5}{v2,vf}
\fmf{fermion}{vf,o5}
\fmf{phantom}{i1,i2}
\fmffreeze
%\fmf{boson,tension=1.5}{v1,v3}
%\fmf{boson}{v3,v2}
%\fmffreeze
%\fmf{plain}{v3,vh}
%\fmf{phantom,tension=1.1}{vh,vb}
%\fmf{photon}{o4,vh,o3}
%\fmf{phantom}{o3,vb}
%\fmf{photon}{vh,o3}

\end{fmfgraph}
%\end{fmffile}

%\begin{fmfchar*}(40,25)%(x.y) sono la grandezza del diagramma
%\fmfleft{i1} \fmfright{o1,o2}
%\fmf{photon}{i1,v4}
%\fmf{quark}{o1,v1,v2,v3,v4,v5,v6,v7,o2}
%\fmffreeze
%\fmf{gluon}{v1,v7}
%\fmf{gluon}{v2,v6}
%\fmf{gluon}{v3,v5}


%\fmf{plain,lab=$T$,lab.side=right}{v3,v2}
%\fmf{plain,lab=$T$,lab.side=right}{v2,v4} 

%infine piazzo le linee esterne
%\fmf{plain}{p1,v3}
%\fmfv{label=$t$,label.dist=0.02w,label.angle=-40}{v3} %
%\fmf{dashes}{v3,pHiggs}
%\fmf{dashes,label=\tiny$H$,label.side=left}{pHiggs,p2}%questo è Higgs

%\fmf{plain}{v4,p3}% questo è il top
%\fmfv{label=$t$,label.dist=0.02w,label.angle=40}{v4}
%\fmf{dashes}{v4,pHiggs2}
%\fmf{dashes,label=\tiny$H$,label.side=right}{pHiggs2,p4}% questo è l'Higgs


%\fmf{photon}{o1,pphot}
%\fmf{photon}{pphot,p4}%fotone
%\fmf{photon}{p4,pphot2}
%\fmf{photon}{pphot2,o2}%fotone

%\fmf{plain}{p3,o3}%b
%\fmf{plain}{p3,pw}%W
%\fmfv{lab=\tiny$W$,lab.dist=0.01w,l.a=120}{pw}%sotto
%\fmf{plain}{o5,pw,o4}%q,q

%\fmf{plain}{p1,o6}
%\fmf{plain}{p1,pw2}

%\fmf{plain}{o8,pw2,o7}
%\fmfv{lab=\tiny$W$,lab.dist=0.02w,l.a=180}{pw2} %sopra

%\fmf{plain}{o10,p2,o9}



%\end{fmfchar*}
\end{fmffile}

\end{document}