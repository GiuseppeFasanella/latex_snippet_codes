\documentclass{article}
\usepackage{feynmp}
\usepackage{color}
\pagestyle{empty} %Takes away pagenumbers etc.
\begin{document}

\unitlength = 1mm
\begin{fmffile}{thq_diag} 
\begin{fmfchar*}(80,50)%(x.y) sono la grandezza del diagramma
\fmfleft{i1,i2}\fmflabel{$b$}{i1}\fmflabel{$q$}{i2}
\fmfright{o1,o2,o3,o4,o5} \fmflabel{$b$}{o1} \fmflabel{$W$}{o2}\fmflabel{$\gamma$}{o3}\fmflabel{$\gamma$}{o4}
\fmflabel{$q^{'}$}{o5}
\unitlength = 1mm
\fmfset{arrow_ang}{10} %il default dello spessore delle frecce è 15 ma è troppo grosso

%\boldmath$b$
%comincia a chiudere le linee e costruire il diagramma
%\fmf{fermion}{i2,v2,o5}
%\fmf{fermion}{i1,v1,v3}
%\fmf{fermion,label=$t$,label.side=right}{v3,v4}
%\fmf{fermion}{v4,o1}
%\fmf{boson,label=$W$,label.side=right}{v1,v2}
%\fmf{dashes,label=$H$,label.side=left}{v3,v5} %questo è l'Higgs
%\fmf{photon}{o4,v5}
%\fmf{photon}{v5,v6}
%\fmf{photon}{v6,o3}%giusto per allungare un po' la gamba del fotone
%\fmf{boson}{v4,o2} 
%
%\fmfdot{v5}% questo decide quali vertici abbiamo un punto marcato

\fmf{fermion}{i2,v2,o5}
%\fmf{plain}{i2,v2,v7}
%\fmf{fermion}{v7,o5}
\fmf{fermion}{i1,v1,v3}
\fmf{fermion,label=$t$,label.side=right}{v3,v4}
\fmf{fermion}{v4,o1}
\fmf{boson,label=$W$,label.side=right}{v1,v2}
\fmf{dashes,label=$H$,label.side=left}{v3,v5} %questo è l'Higgs
\fmf{photon}{o4,v5}
\fmf{photon}{v5,v6}
\fmf{photon}{v6,o3}%giusto per allungare un po' la gamba del fotone
\fmf{boson}{v4,o2} 

%\fmfdot{v5}% questo decide quali vertici abbiamo un punto marcato
\fmfv{decor.shape=circle,decor.filled=full, decor.size=3thick}{v5}%cosi' setti pure la grandezza del blob

\fmffreeze%calcola posizioni dei vertici
\fmfshift{10,0}{v1}%shifta posizioni dei vertici, c'e' anche force per forzare la posizione



\end{fmfchar*}
\end{fmffile}

\end{document}
