%Supponiamo vuoi fare un indice dei nomi e dei luoghi.

%1. All'inizio del documento
\usepackage{index}
%Per ogni indice che vuoi avere devi specificare
%{nome} {estensione di prima compilazione} {estensione finale}
%E' importante che ogni indice abbia una diversa estensione, sia di prima compilazione che finale
%A parte il default non credo ci siano regole per le estensioni degli indici successivi al primo
%Io mi sono inventato ndx -> nnd e tdx->tnd ma non credo esista una regola in merito
\newindex{default}{idx}{ind}{Indice dei nomi}
\newindex{luoghi}{ndx}{nnd}{Indice dei luoghi}
\newindex{fatti}{tdx}{tnd}{Indice dei fatti}


%2. All'interno del documentoemas
Pascal\index{Pascal, B.}
Madagascar\index[luoghi]{Madagascar}
Rivoluzione francese\index[fatti]{Rivoluzione francese}


%3. Alla fine del documento
\printindex %questo sarebbe il default
\printindex[luoghi] %questo sarebbe l'indice dei  luoghi
\end{document}

%4. Alla compilazione
pdflatex documento_breve %avrai gli indici in estensione .idx, .ndx, .tdx (quelle che ho chiamato di prima compilazione)
makeindex documento_breve %compila il primo indice (il default) da .idx a .ind (c'e' anche un log file .ilg)
makeindex -o documento_breve.nnd -t documento_breve.nlg documento_breve.ndx %occupati ora del secondo
makeindex -o documento_breve.tnd -t documento_breve.tlg documento_breve.tdx %e anche del terzo indice
pdflatex documento_breve %ricompila tutto
pdflatex documento_breve
