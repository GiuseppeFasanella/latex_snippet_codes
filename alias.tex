\def    \mee               {\mbox{$M_{ee}$} } %Nel  testo scrivi solo \mee
\def    \pt                {\mbox{$p_{\mathrm{T}}$ }}
%accenti
\def \E {\`E}
\def \e {è} %e aperta                                                                                              
\def \ee {é} %e chiusa                                                             
\def \A {\`A}
\def \a {\`a}
\def \i {\`i}
\def \u {\`u}
\def \o {\`o}

%%% Per usare gli alias fai cosi': (dopo ogni alias metti uno spazio esplicito nel sorgente latex)                                                                                                  
%\E~stato bello per\o~sai \e~difficile perch\ee~la vita \e~dura.                                                                     
%La motivazione per la quale non inserisco lo spazio nell'alias stesso e' perche' a volte voglio inserire una virgola:    
% ad esempio devo garantarmi la possibilita' di scrivere:  la realtà è, forse                                               
% Va scritto in questo modo: la realt\a~\e,~forse; ora se la definizione di \e comprendesse lo spazio
% otterrei invece:  la realtà è , forse. (un brutto spazio dopo la e aperta).
